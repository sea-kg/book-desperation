\newpage
\chapter{Шпион}
{\centering
	September 1, 2013 - September 22, 2014\par
}
\par
Закончилась военная подготовка… пора было отправляться
на службу в дальний космос. Где то там шла очень долгая война.
Эрик получил свое первое назначение в спец-полк.
Кто то поговаривал что спец-полки - просто фарш который отдается
противнику просто так. Другие говорили что они герои… в любом случае
никто не возвращался при жизни текущего поколения да и первый засланный
полк сможет вернуться только через лет 500.
\par
Эрик, как и другие, был воспитан в общественных учреждениях и
ничем не отличался. Там всех учат не задавать вопросов, говорить мало,
не обсуждать ничего кроме своих потребностей. Из них готовили боевых машин
из многих других в этом мире империи Дрокш.
\par
Если одним быстрым взглядом охватить жизнь людей в этот период то может
показаться что их жизнь серая и унылая и не стоит того внимания.
Но если заглянуть за занавеску или напроститься в гости, то вы не найдете
злых и червствых людей - все они очень добры и приветливы.
Они ценят моменты жизни в которых можно просто общаться быть собой,
потому что в другое время приходиться быть "винтиками" системы, огромной военной машины
находящейся в состоянии войны.
\par
Первые дни Эрик превел в анабиозе. Его корабль летел сквозь пространство 
на скорости равной скорости света уносившей всех его немногих знакомых и друзей
в тень прошлого. Оставляя только небольшой отпечаток в его памяти как воробей на 
снегу, и этот след будет уничтожен как придет весна. Может это такое сложное время,
а может таков человек но новая история вытесняет старую, переписывая ее во все более
сжатые главы далекого прошлого.
\par
Прошло 103 световых года в анабиозе.
Пробуждение из анабиоза началось еще на подлете к их первому укреплению на территории врага.
Враг еще не пронюхал где их база. Это была маленькая планета "Хромус" названная так в честь
хитрого полководца из далекого 22 столетия, от которого остались только легенды
и некоторые принципы стратегических приемов во время ведения войны с неизвестным противником.
\par
При подлете к планете спец-полк был во все оружии, так как враг мог появиться вблизи
и уничтожить планету за 30 секунд. После высадки все прошли курс реабилитации
после анабиозного сна, медицинское обследование а также курс прививок от болезней нового мира.
\par
Дальше практически без перерывов, только на сон и еду, начались тренировки.
Изнурительные физические, психологические а также обучение. 
Физичиские тренировки - наверное самый древний вид. Психологические - тоже достаточно старый,
его суть заключается в том что бы не терять психологическое равновесие
при различных обстоятельствах. А обучением закладывались те основы
при которых любой мог заменить любого. Начиная от полевого врача и заканчивая 
техником. Конечно их знаний не будет хватать не будет хватать и опыта но это лучше
чем потерять единственного врача в бою и ждать замены еще несколько сотен лет.
Каждый солдат должен просто уметь все.
\par
В тернировках теряется счет времени и при таких интенсивных тренировках некогда общаться.
Поэтому никто незнает кто и откуда. Все знают только одно - они новички.
Так продолжалось неделю пока...
\par
Прерванный сон, звоном красной тревогой… подъем… есть только 30 секунд и
10 в запасе для того что бы одеться взять оружие и спуститься в кабину
вместе с соседом и остальными бритоголовыми членами постоянной армии империи.
Кабина дальше перекинет весь полк на корабль и еще дальше в глубокий
космос до новой базы. Так как с этой придется попрощаться. Бегство - это
единственный тактический ход что бы сохранить армию и узнать на что
способен противник. Уже через несколько часов островокбудет полностью
стерт с лица Хромсу. Этот островок давал полку №315 покой целую неделю.
\par
За 300 лет очень сильно изменилась тактика ведения войск. Первые войны
были построены только по одной стратегии: защита и нападения… потом через
50 лет наступила время тактики запугивания - когда с помощью мощного оружия
можно уничтожить своего противника несколько сот раз но вот не задача
можно подпортить его территории, что сам воспользоваться не сможешь -
очень не выгодно, но зато можно мелких тыркать и забирать
под “честным” предлогом. 
\par
Но сейчас войны изменились очень сильно. Практически стычек нет, об
атаке противник узнает или от шпионов… или от техников. Когда то давно
когда была первая межпланетная война и один известный полководец Ормел
сказал, что шпионы могут обеспечить победу малой кровью с обоих сторон,
из за того что войну начинает несколько человек и если их
дескредитировать то война может закончиться не начавшись.
Но теперь не только шпионы могли предупредить атаку противника.
Также существуют технические средства, наподобие дальних сенсоров,
“поплавков” а также  компьютерных прогнозов и прочих технических
штуковин похожих на магию, в которых не так то и много людей разбирается.
Правда называют это высоко технологически сложно решаемые задачи.
\par
Но вернемся к полку “315”, его задача была измотать противника и узнать
слабые стороны. Этот полк все время находился на вражеской территории
и а также должен был отвлечь от некоторых действий империи...
\par
