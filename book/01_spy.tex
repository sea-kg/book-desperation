\newpage
\section{Шпион}
{\centering
	September 1, 2013 - September 22, 2014\par
}
\par
Закончилась школьно-военная подготовка… пора было отправляться
на службу в дальний космос, где то там шла война.
Эрик получил назначение в спецполк в самом начале своего срока службы.
Кто то поговаривал что эти полки - просто фарш который отдается
противнику просто так. Другие говорили что они герои… в любом случае
никто не возвращался при жизни текущего поколения да и первый засланный
полк сможет вернуться только через лет 500. 
\par
Эрик, как и другие, по своему воспитанию в общественных учреждениях
ничем не отличался. Там всех учат не задавать вопросов, говорить мало,
не обсуждать ничего кроме своих потребностей. То есть они были
практически машинами как и многие многие другие в этом большом мире империи Дрокнш.
\par
Прерванный сон, звоном красной тревогой… подъем… есть только 30 секунд и
10 в запасе для того что бы одеться взять оружие и спуститься в кабину
вместе с соседом и остальными бритоголовыми членами постоянной армии империи.
Кабина дальше перекинет весь полк на корабль и еще дальше в глубокий
космос до новой базы. Так как с этой придется попрощаться. Бегство - это
единственный тактический ход что бы сохранить армию и узнать на что
способен противник. Уже через несколько часов островокбудет полностью
стерт с лица Хромсу. Этот островок давал полку №315 покой целую неделю.
\par
За 300 лет очень сильно изменилась тактика ведения войск. Первые войны
были построены только по одной стратегии: защита и нападения… потом через
50 лет наступила время тактики запугивания - когда с помощью мощного оружия
можно уничтожить своего противника несколько сот раз но вот не задача
можно подпортить его территории, что сам воспользоваться не сможешь -
очень не выгодно, но зато можно мелких тыркать и забирать
под “честным” предлогом. 
\par
Но сейчас войны изменились очень сильно. Практически стычек нет, об
атаке противник узнает или от шпионов… или от техников. Когда то давно
когда была первая межпланетная война и один известный полководец Ормел
сказал, что шпионы могут обеспечить победу малой кровью с обоих сторон,
из за того что войну начинает несколько человек и если их
дескредитировать то война может закончиться не начавшись.
Но теперь не только шпионы могли предупредить атаку противника.
Также существуют технические средства, наподобие дальних сенсоров,
“поплавков” а также  компьютерных прогнозов и прочих технических
штуковин похожих на магию, в которых не так то и много людей разбирается.
Правда называют это высоко технологически сложно решаемые задачи.
\par
Но вернемся к полку “315”, его задача была измотать противника и узнать
слабые стороны. Этот полк все время находился на вражеской территории
и а также должен был отвлечь от некоторых действий империи...
\par
